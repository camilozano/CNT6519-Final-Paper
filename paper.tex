
%% The first command in your LaTeX source must be the \documentclass command.
\documentclass[acmlarge]{acmart}

\usepackage{lipsum}
 
%% \BibTeX command to typeset BibTeX logo in the docs
\AtBeginDocument{%
  \providecommand\BibTeX{{%
    \normalfont B\kern-0.5em{\scshape i\kern-0.25em b}\kern-0.8em\TeX}}}

%% Rights management information.  This information is sent to you
%% when you complete the rights form.  These commands have SAMPLE
%% values in them; it is your responsibility as an author to replace
%% the commands and values with those provided to you when you
%% complete the rights form.
 \setcopyright{acmcopyright}
 \copyrightyear{2021}
 \acmYear{2021}
 \acmDOI{}


%%
%% These commands are for a JOURNAL article.
 \acmJournal{POMACS}
 \acmVolume{1}
 \acmNumber{1}
 \acmArticle{111}
 \acmMonth{11}

%%
%% Submission ID.
%% Use this when submitting an article to a sponsored event. You'll
%% receive a unique submission ID from the organizers
%% of the event, and this ID should be used as the parameter to this command.
%%\acmSubmissionID{123-A56-BU3}


\begin{document}


\title{Wireless Forensics Principles and Investigations}



\author{Evan Lomax}
\email{evn.lmx@gmail.com}
\affiliation{%
  \institution{University of Central Florida}
  \city{Orlando}
  \country{USA}
}

\author{Shenequa Hayes}
\email{shenequa.hayes@gmail.com}
\affiliation{%
  \institution{University of Central Florida}
  \city{Orlando}
  \country{USA}
}
\author{Camilo Lozano}
\email{clozano@knights.ucf.edu}
\affiliation{%
  \institution{University of Central Florida}
  \city{Orlando}
  \country{USA}
}
\author{Shannon Padilla}
\email{JPCharme@gmail.com}
\affiliation{%
  \institution{University of Central Florida}
  \city{Orlando}
  \country{USA}
}



%% By default, the full list of authors will be used in the page
%% headers. Often, this list is too long, and will overlap
%% other information printed in the page headers. This command allows
%% the author to define a more concise list
%% of authors' names for this purpose.
\renewcommand{\shortauthors}{Lomax, et al.}

%%
%% The abstract is a short summary of the work to be presented in the
%% article.
\begin{abstract}
We will discuss the current and emerging principles and investigations as it pertains to wireless digital forensics. We will look at the past, the present and future of wireless forensics. We will examine the major sub-branches, to include: 802.11 network forensics, IoT vulnerabilities, mobile forensics, available wireless security and forensics software tools. We will compare and contrast some of the major forensic tool suites as well as some emerging technologies. 
\end{abstract}

%%
%% The code below is generated by the tool at http://dl.acm.org/ccs.cfm.
%% Please copy and paste the code instead of the example below.
%%
\begin{CCSXML}
<ccs2012>
   <concept>
       <concept_id>10003033.10003083.10003014</concept_id>
       <concept_desc>Networks~Network security</concept_desc>
       <concept_significance>500</concept_significance>
       </concept>
   <concept>
       <concept_id>10002978.10003014.10003017</concept_id>
       <concept_desc>Security and privacy~Mobile and wireless security</concept_desc>
       <concept_significance>500</concept_significance>
       </concept>
   <concept>
       <concept_id>10002978.10002979.10002983</concept_id>
       <concept_desc>Security and privacy~Cryptanalysis and other attacks</concept_desc>
       <concept_significance>300</concept_significance>
       </concept>
 </ccs2012>
\end{CCSXML}

\ccsdesc[500]{Networks~Network security}
\ccsdesc[500]{Security and privacy~Mobile and wireless security}
\ccsdesc[300]{Security and privacy~Cryptanalysis and other attacks}

%%
%% Keywords. The author(s) should pick words that accurately describe
%% the work being presented. Separate the keywords with commas.
\keywords{networks, security, IOT, 802.11}


%%
%% This command processes the author and affiliation and title
%% information and builds the first part of the formatted document.
\maketitle

\section{Introduction}
\lipsum[1]

\section{Internet of Things}

A new growing field of devices has sprung up recently with the development and advances of the field of smart homes. These devices usually feature some sort of network connectivity either Wi-Fi or Bluetooth and are considered a form of edge computing. IoT or Internet of Things has been defined as connectable devices and sensors that allow for remote monitoring and manipulation. In a way they are considered a form of edge computing. The biggest rise in these IoT devices has been in the consumer applications smart home sector where it has really taken off in the years between 2008 and 2009 \cite{evans_iot_2011}.

The ever increasing presence of these IoT devices present a new concern as they are a prone vector for security vulnerabilities. Specially so because of the preferences for manufacturers and consumers to turn to a Wi-Fi internet solution for connecting their devices to outside. With little to no regulation, or even certifications, on how these devices can be implemented it has left a big hole in the market making it difficult to know what to trust. 
% Summary for this section might edit later
In this section we will talk about known big security exploits that have already occurred, best practices to follow, and talk about existing forensic methods that can be used to investigate these vulnerabilities.

\subttitle{Background}

One of the big issues mentioned is the low level of security that is present in the IoT devices. One of the reasons they are such a threat is the current trend for these devices is to be individually connected to the internet. Initial implementations of smart home appliances relied more on a hub model where the devices were connected to a central hub. This meant that the devices were not connected to the internet and therefore had no way to communicate with the outside world except through the hub. The idea is to just expose one device to the internet and attempt to secure that one as much as possible. With the rise of cheaper microcontroller boards that had built in Wi-Fi, specially the ESP8266, the standard model changed and it becaome more common to have each individual device connected to the main router. 






\bibliographystyle{ACM-Reference-Format}
\bibliography{references}



\end{document}
\endinput
%%
