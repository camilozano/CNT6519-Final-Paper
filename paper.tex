
%% The first command in your LaTeX source must be the \documentclass command.
\documentclass[acmlarge]{acmart}

\usepackage{lipsum}
 
%% \BibTeX command to typeset BibTeX logo in the docs
\AtBeginDocument{%
  \providecommand\BibTeX{{%
    \normalfont B\kern-0.5em{\scshape i\kern-0.25em b}\kern-0.8em\TeX}}}

%% Rights management information.  This information is sent to you
%% when you complete the rights form.  These commands have SAMPLE
%% values in them; it is your responsibility as an author to replace
%% the commands and values with those provided to you when you
%% complete the rights form.
 \setcopyright{acmcopyright}
 \copyrightyear{2021}
 \acmYear{2021}
 \acmDOI{}


%%
%% These commands are for a JOURNAL article.
 \acmJournal{POMACS}
 \acmVolume{1}
 \acmNumber{1}
 \acmArticle{111}
 \acmMonth{11}

%%
%% Submission ID.
%% Use this when submitting an article to a sponsored event. You'll
%% receive a unique submission ID from the organizers
%% of the event, and this ID should be used as the parameter to this command.
%%\acmSubmissionID{123-A56-BU3}


\begin{document}


\title{Wireless Forensics Principles and Investigations}



\author{Evan Lomax}
\email{evn.lmx@gmail.com}
\affiliation{%
  \institution{University of Central Florida}
  \city{Orlando}
  \country{USA}
}

\author{Shenequa Hayes}
\email{shenequa.hayes@gmail.com}
\affiliation{%
  \institution{University of Central Florida}
  \city{Orlando}
  \country{USA}
}
\author{Camilo Lozano}
\email{clozano@knights.ucf.edu}
\affiliation{%
  \institution{University of Central Florida}
  \city{Orlando}
  \country{USA}
}
\author{Shannon Padilla}
\email{JPCharme@gmail.com}
\affiliation{%
  \institution{University of Central Florida}
  \city{Orlando}
  \country{USA}
}



%% By default, the full list of authors will be used in the page
%% headers. Often, this list is too long, and will overlap
%% other information printed in the page headers. This command allows
%% the author to define a more concise list
%% of authors' names for this purpose.
\renewcommand{\shortauthors}{Lomax, et al.}

%%
%% The abstract is a short summary of the work to be presented in the
%% article.
\begin{abstract}
We will discuss the current and emerging principles and investigations as it pertains to wireless digital forensics. We will look at the past, the present and future of wireless forensics. We will examine the major sub-branches, to include: 802.11 network forensics, IoT vulnerabilities, mobile forensics, available wireless security and forensics software tools. We will compare and contrast some of the major forensic tool suites as well as some emerging technologies. 
\end{abstract}

%%
%% The code below is generated by the tool at http://dl.acm.org/ccs.cfm.
%% Please copy and paste the code instead of the example below.
%%
\begin{CCSXML}
<ccs2012>
   <concept>
       <concept_id>10003033.10003083.10003014</concept_id>
       <concept_desc>Networks~Network security</concept_desc>
       <concept_significance>500</concept_significance>
       </concept>
   <concept>
       <concept_id>10002978.10003014.10003017</concept_id>
       <concept_desc>Security and privacy~Mobile and wireless security</concept_desc>
       <concept_significance>500</concept_significance>
       </concept>
   <concept>
       <concept_id>10002978.10002979.10002983</concept_id>
       <concept_desc>Security and privacy~Cryptanalysis and other attacks</concept_desc>
       <concept_significance>300</concept_significance>
       </concept>
 </ccs2012>
\end{CCSXML}

\ccsdesc[500]{Networks~Network security}
\ccsdesc[500]{Security and privacy~Mobile and wireless security}
\ccsdesc[300]{Security and privacy~Cryptanalysis and other attacks}

%%
%% Keywords. The author(s) should pick words that accurately describe
%% the work being presented. Separate the keywords with commas.
\keywords{networks, security, IOT, 802.11}


%%
%% This command processes the author and affiliation and title
%% information and builds the first part of the formatted document.
\maketitle

\section{Introduction}
%% TODO add introduction

In this paper we seek to provide a comprehensive overview of the current and emerging wireless security and forensic techniques used today. Wireless forensics is a vast and complex field with a rising demand for expert forensics and security professionals. By taking a deep dive in the field of 802.11 network forensics, IoT vulnerabilities, mobile forensics, and software forensics 

\section{Internet of Things}

A new growing field of devices has sprung up recently with the development and advances of the field of smart homes. These devices usually feature some sort of network connectivity either Wi-Fi or Bluetooth and are considered a form of edge computing. IoT or Internet of Things has been defined as connectable devices and sensors that allow for remote monitoring and manipulation. In a way they are considered a form of edge computing. The biggest rise in these IoT devices has been in the consumer applications smart home sector where it has really taken off in the years between 2008 and 2009 \cite{evans_iot_2011}.

The ever increasing presence of these IoT devices present a new concern as they are a prone vector for security vulnerabilities. Specially so because of the preferences for manufacturers and consumers to turn to a Wi-Fi internet solution for connecting their devices to outside. With little to no regulation, or even certifications, on how these devices can be implemented it has left a big hole in the market making it difficult to know what to trust. 
% Summary for this section might edit later
In this section we will talk about known big security exploits that have already occurred, best practices to follow, and talk about existing forensic methods that can be used to investigate these vulnerabilities.


\subsection{Background}

One of the big issues mentioned is the low level of security that is present in the IoT devices. One of the reasons they are such a threat is the current trend for these devices is to be individually connected to the internet. Initial implementations of smart home appliances relied more on a hub model where the devices were connected to a central hub. This meant that the devices were not connected to the internet and therefore had no way to communicate with the outside world except through the hub. The idea is to just expose one device to the internet and attempt to secure that one as much as possible. With the rise of cheaper micro controller boards that had built in Wi-Fi, specially the ESP8266, the standard model changed and it became more common to have each individual device connected to the main access point. One of the biggest benefits to this model is also one of its drawbacks; allowing consumers to have multiple different ecosystems. This means that there is even more potential for security attacks as each device can be from a different vendor that may or may not be secured.  

The problem becomes that in a race to the bottom of prices it has lead to corner cutting on the software side to just serve the minimum purpose without insuring any protection. We'll follow up with a couple of known intrusions that have occurred ranging from small mistakes to major security concerns. 


\subsection{Tapplock Smart Lock}

In 2018 smart device manufacturer Tapplock began selling a \$100 smart lock, mean to be a sort of gym locker lock that would allow for unlocking with a fingerprint, with a solid steel construction it also had a mobile app that would allow for unlocking the lock over Bluetooth Low Energy (BLE).
On first glance the lock boasted an 'AES 128-bit encryption', but forensic experts quickly started finding some flaws in its implementations. According to the team, they went from obtaining the lock to completely defeating its security in a mere 45 minutes! \cite{Ken_2018}

What they did was manually pair the BLE device to their computer in which they could then use a tool like Wireshark to sniff the traffic between the lock. To their surprise the lock was using HTTP API requests to control the lock over plaintext. This was the first major red flag, this means anyone able to intercept or be in the middle of the communication would be able to sniff and replicate packets to control it. 

The way they were 'securing' their requests was by using some random data attached to the request. The problem was that it was the same random data on each request, meaning that it was a symmetric key. Knowing this allows can allow for any payload for it to be sent.
Additionally another feature of the app allows for the user to share temporarily access to a guest, but as it turns out the key that you share is the same master key as the user. So even though you can set an expiration date for the guest, they essentially have access to the whole security.

The biggest revelation came from a dumping of code on the chip which revealed how the encryption key was being generated. It generates a hash based of some seed value, which turns out is just the MAC address as a string of the BLE devices, which is the only piece of data that is available to the an attacker!
This was then automated to be used in an app which would scan for nearby BLE devices and identify those that were a Tapplock device then use its MAC address to generate a key and unlock the key. Meaning that anyone with a phone in range of a Tapplock device can unlock it in less than a a couple seconds!

\subsection{Mirai Malware}

One of the most common type of internet based attacks is a Denial-of-Service (DoS) attack. This is when a malicious user or group of users is able to flood a server with requests and cause it to crash. Usually the way to prevent his is by having a firewall in place that blocks all traffic from the attacker. This is done by identifying some sort of pattern in the traffic and blocking it. An advanced form of this attack is a Distributed DoS (DDoS), which consists of a coordinated attacked by many with varying ranging IPs that would be much harder to identify and block. 

In 2016 a botnet malware started to randomly appear. A large amount of CCTV, routers and other IoT devices were being compromised by this malware. It was specially hard to detect as the IoT devices remained mostly fine except for every now and them some sluggishness every and increased bandwith. What was interesting is that this malware is that it is not complex in its nature. What it does once a device is infected is randomly scan for IPs and attempt to ping them on certain known ports. The developer compiled a list of known manufacturers default ports, usernames and passwords. Each node then just attempts pinging until it finds an open device, then its IP is uploaded to a controller that then contacts the other nodes to try to authenticate on the known device until eventually some are compromised by their default passwords. 

Once inside on a compromised device, the malware installs itself on the new device and continues to propagate. This allowed the malware to scale massively. This was then used to DDoS a bunch of websites, including a OVH (one of France's largest hosting companies), Dyn (a DNS provider), and even a cyber security journalist. At its peak there was a recorded bandwidth hitting 1 Tbit/s!\cite{Hertzberg_Mirai_2016}

Most interestingly the was that the source code of the malware was eventually leaked and the creator was identified! Apparently the author had a IT security company that offered protection against DDoS. In a way they were running a protection racket, initially starting with Minecraft servers.

This attack revealed how major manufacturers were using default credentials and allowing easy exploitable security issues. Since the code has been open sourced others have taken to fork it and adapt the same concept to different devices.


\bibliographystyle{ACM-Reference-Format}
\bibliography{references}



\end{document}
\endinput
%%
